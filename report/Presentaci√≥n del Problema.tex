% !TEX root = main.tex
\section{Presentación del problema}
La aseguradora Corredores Argentinos S.A. se dispone a crear un programa de formación para sus empleados, con el objetivo de mejorar sus competencias en el área de seguros y gestión comercial. Para ello, se ha diseñado un calendario académico que contempla la asignación de materias, profesores y cohortes, respetando las correlatividades y restricciones logísticas propias de la organización.

A continuación se detallan los elementos principales del problema:

\subsection*{Materias}
\begin{table}[ht]
\centering
\begin{tabular}{|l|}
\hline
	extbf{Materias} \\
\hline
Intro a Seguros \\
Matemática Actuarial \\
Riesgos \\
Legislación \\
Seguros de Vida \\
Seguros Generales \\
Gestión Comercial \\
Reaseguros \\
\hline
\end{tabular}
\caption{Listado de materias del programa}
\end{table}

\subsection*{Profesores}
\begin{table}[ht]
\centering
\begin{tabular}{|l|}
\hline
	–extbf{Profesores} \\
\hline
Prof. A \\
Prof. B \\
Prof. C \\
Prof. D \\
\hline
\end{tabular}
\caption{Profesores asignados al programa}
\end{table}

\subsection*{Asignación de profesores a materias}
\begin{table}[ht]
\centering
\begin{tabular}{|l|l|}
\hline
	extbf{Materia} & \textbf{Profesor} \\
\hline
Intro a Seguros & Prof. A \\
Matemática Actuarial & Prof. B \\
Riesgos & Prof. C \\
Legislación & Prof. D \\
Seguros de Vida & Prof. A \\
Seguros Generales & Prof. B \\
Gestión Comercial & Prof. C \\
Reaseguros & Prof. D \\
\hline
\end{tabular}
\caption{Asignación fija de profesores a materias}
\end{table}

\subsection*{Cohortes}
\begin{table}[ht]
\centering
\begin{tabular}{|l|}
\hline
	extbf{Cohortes} \\
\hline
C1 \\
C2 \\
C3 \\
C4 \\
C5 \\
\hline
\end{tabular}
\caption{Cohortes de empleados participantes}
\end{table}

\subsection*{Correlatividades}
\begin{table}[ht]
\centering
\begin{tabular}{|l|l|}
\hline
	extbf{Materia} & \textbf{Correlativa} \\
\hline
Matemática Actuarial & Intro a Seguros \\
Riesgos & Matemática Actuarial \\
Legislación & Intro a Seguros \\
Seguros de Vida & Riesgos \\
Seguros Generales & Legislación \\
Gestión Comercial & Seguros Generales \\
Reaseguros & Seguros de Vida \\
\hline
\end{tabular}
\caption{Correlatividades entre materias}
\end{table}

El objetivo es construir un calendario óptimo que asigne las materias a los distintos cuatrimestres, días y turnos, respetando las correlatividades, la disponibilidad de profesores y las restricciones logísticas, como la cantidad máxima de clases por día y la prohibición de ciertos bloques horarios.