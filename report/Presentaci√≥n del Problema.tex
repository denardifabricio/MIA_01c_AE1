% !TEX root = MIA_c01_AE1.tex
\section{Presentación del problema}
La aseguradora Corredores Argentinos S.A. se dispone a crear un programa de formación para sus empleados, con el objetivo de mejorar sus competencias en el área de seguros y gestión comercial. Para ello, se ha diseñado un calendario académico que contempla la asignación de materias, profesores y cohortes, respetando las correlatividades y restricciones logísticas propias de la organización.

A continuación se detallan los elementos principales del problema:

\subsection*{Repositorio del proyecto}
El código fuente completo de este proyecto, incluyendo la implementación del algoritmo de optimización, se encuentra disponible en el repositorio de GitHub:

\texttt{https://github.com/denardifabricio/MIA\_01c\_AE1}

Este repositorio contiene todos los archivos necesarios para ejecutar el proyecto, incluyendo los datos de entrada, la implementación del algoritmo PSO, y los scripts para generar los calendarios y visualizaciones.

\subsection*{Materias}
\begin{table}[ht]
\centering
\begin{tabular}{|l|}
\hline
	extbf{Materias} \\
\hline
Intro a Seguros \\
Matemática Actuarial \\
Riesgos \\
Legislación \\
Seguros de Vida \\
Seguros Generales \\
Gestión Comercial \\
Reaseguros \\
\hline
\end{tabular}
\caption{Listado de materias del programa}
\end{table}

\subsection*{Profesores}
\begin{table}[ht]
\centering
\begin{tabular}{|l|}
\hline
	–extbf{Profesores} \\
\hline
Prof. A \\
Prof. B \\
Prof. C \\
Prof. D \\
\hline
\end{tabular}
\caption{Profesores asignados al programa}
\end{table}

\subsection*{Asignación de profesores a materias}
\begin{table}[ht]
\centering
\begin{tabular}{|l|l|}
\hline
	extbf{Materia} & \textbf{Profesor} \\
\hline
Intro a Seguros & Prof. A \\
Matemática Actuarial & Prof. B \\
Riesgos & Prof. C \\
Legislación & Prof. D \\
Seguros de Vida & Prof. A \\
Seguros Generales & Prof. B \\
Gestión Comercial & Prof. C \\
Reaseguros & Prof. D \\
\hline
\end{tabular}
\caption{Asignación fija de profesores a materias}
\end{table}

\subsection*{Cohortes}
\begin{table}[ht]
\centering
\begin{tabular}{|l|}
\hline
	extbf{Cohortes} \\
\hline
C1 \\
C2 \\
C3 \\
C4 \\
C5 \\
\hline
\end{tabular}
\caption{Cohortes de empleados participantes}
\end{table}

\subsection*{Correlatividades}
\begin{table}[ht]
\centering
\begin{tabular}{|l|l|}
\hline
	extbf{Materia} & \textbf{Correlativa} \\
\hline
Matemática Actuarial & Intro a Seguros \\
Riesgos & Matemática Actuarial \\
Legislación & Intro a Seguros \\
Seguros de Vida & Riesgos \\
Seguros Generales & Legislación \\
Gestión Comercial & Seguros Generales \\
Reaseguros & Seguros de Vida \\
\hline
\end{tabular}
\caption{Correlatividades entre materias}
\end{table}

El objetivo es construir un calendario óptimo que asigne las materias a los distintos cuatrimestres, días y turnos, respetando las correlatividades, la disponibilidad de profesores y las restricciones logísticas, como la cantidad máxima de clases por día y la prohibición de ciertos bloques horarios.

\subsection{Restricciones del problema}
El problema de programación de calendarios académicos presenta diversas restricciones que deben ser consideradas para obtener una solución factible y de calidad. Estas restricciones se clasifican en fuertes (hard constraints) y débiles (soft constraints) según su nivel de criticidad:

\subsubsection{Restricciones Fuertes (Hard Constraints)}
Las restricciones fuertes son aquellas que \textbf{no pueden ser violadas} bajo ninguna circunstancia, ya que comprometen la factibilidad de la solución:

\begin{itemize}
    \item \textbf{Correlatividades (Prerequisites)}: Las materias deben respetar estrictamente el orden de correlatividades. Una materia no puede ser dictada en un cuatrimestre igual o anterior al de su correlativa.
    \item \textbf{Conflictos de horarios}: No puede haber solapamiento de horarios para un mismo profesor o cohorte en un mismo día y turno.
    \item \textbf{Bloques horarios prohibidos}: Ciertos días y turnos están bloqueados y no pueden ser utilizados para dictar clases.
\end{itemize}

\subsubsection{Restricciones Débiles (Soft Constraints)}
Las restricciones débiles pueden ser violadas, pero con penalizaciones en la función objetivo que afectan la calidad de la solución:

\begin{itemize}
    \item \textbf{Distribución equilibrada por cuatrimestre}: Se busca una distribución balanceada de materias entre los diferentes cuatrimestres del programa.
    \item \textbf{Sobrecarga de profesores}: Los profesores no deberían tener más del máximo permitido de materias por día.
    \item \textbf{Sobrecarga de cohortes}: Las cohortes no deberían tener más del máximo permitido de materias por día.
    \item \textbf{Balance en el uso de días}: Se busca un uso equilibrado de todos los días de la semana disponibles.
    \item \textbf{Uso eficiente de turnos}: Se prefiere el uso de todos los turnos disponibles en cada día.
\end{itemize}

\subsection{Estructura base del problema - Configuración de ejemplo}
A continuación se presenta la estructura completa de configuración utilizada como base para el problema, incluyendo todos los parámetros de restricciones:

\subsubsection{Parámetros generales del programa}
\begin{table}[ht]
\centering
\begin{tabular}{|l|c|l|}
\hline
\textbf{Parámetro} & \textbf{Valor} & \textbf{Descripción} \\
\hline
Duración del programa & 2 años & Extensión total del programa académico \\
Cuatrimestres por año & 2 & División temporal anual \\
Año de inicio & 2026 & Primer año de implementación \\
Clases máximas por semana & 2 & Frecuencia semanal por materia \\
\hline
\end{tabular}
\caption{Configuración temporal del programa}
\end{table}

\subsubsection{Restricciones de capacidad}
\begin{table}[ht]
\centering
\begin{tabular}{|l|c|l|}
\hline
\textbf{Restricción} & \textbf{Límite} & \textbf{Tipo} \\
\hline
Materias por día (profesor) & 2 & Restricción débil \\
Materias por día (cohorte) & 2 & Restricción débil \\
\hline
\end{tabular}
\caption{Límites de capacidad diaria}
\end{table}

\subsubsection{Horarios y turnos disponibles}
\begin{table}[ht]
\centering
\begin{tabular}{|l|l|}
\hline
\textbf{Días de la semana} & \textbf{Turnos disponibles} \\
\hline
Lunes a Viernes & Mañana, Tarde \\
\hline
\textbf{Bloques horarios prohibidos} & \\
\hline
Viernes Tarde & (No disponible) \\
Lunes Mañana & (No disponible) \\
\hline
\end{tabular}
\caption{Disponibilidad horaria del programa}
\end{table}

\subsubsection{Parámetros del algoritmo PSO}
\begin{table}[ht]
\centering
\begin{tabular}{|l|c|l|}
\hline
\textbf{Parámetro PSO} & \textbf{Valor} & \textbf{Descripción} \\
\hline
Número de partículas & 480 & Tamaño del enjambre \\
Iteraciones máximas & 1000 & Criterio de parada \\
Factor cognitivo (c1) & 1.5 & Atracción hacia mejor posición personal \\
Factor social (c2) & 1.5 & Atracción hacia mejor posición global \\
Inercia (w) & 0.7 & Factor de inercia del movimiento \\
\hline
\end{tabular}
\caption{Configuración del algoritmo de optimización PSO}
\end{table}

\subsubsection{Pesos de penalización}
Los pesos de penalización determinan la importancia relativa de cada restricción en la función objetivo:

\begin{table}[ht]
\centering
\begin{tabular}{|l|c|l|}
\hline
\textbf{Tipo de restricción} & \textbf{Peso} & \textbf{Criticidad} \\
\hline
Distribución por cuatrimestre & 20,000 & Alta \\
Correlatividades & 20,000 & Alta \\
Bloques horarios prohibidos & 500 & Media \\
Conflictos de horarios & 500 & Media \\
Sobrecarga de cohortes & 200 & Baja \\
Sobrecarga de profesores & 100 & Baja \\
Uso eficiente de turnos & 100 & Baja \\
Balance de días & 50 & Muy baja \\
\hline
\end{tabular}
\caption{Pesos de penalización por tipo de restricción}
\end{table}

Los pesos más altos corresponden a las restricciones más críticas para el funcionamiento del programa académico. La distribución equilibrada por cuatrimestre y el respeto de las correlatividades tienen la máxima prioridad, mientras que aspectos de optimización como el balance de días tienen menor peso relativo.